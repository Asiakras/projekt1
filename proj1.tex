\documentclass[10pt,a4paper]{article}
\usepackage{graphicx}
\usepackage{subcaption}
\usepackage[polish]{babel}
\usepackage[T1]{fontenc}
\usepackage[utf8]{inputenc}
\usepackage{amsmath, amsfonts, amssymb}
\usepackage{booktabs}
\usepackage[top=2.5cm, bottom=2.5cm, left=2cm, right=2cm]{geometry}
\usepackage{textcomp}
\usepackage{gensymb}
\usepackage{textgreek}
\usepackage{geometry}
\usepackage{pdflscape}
\usepackage{pdfpages}
\usepackage{xcolor}
\usepackage{hyperref}

\newcommand{\authorName}{Joanna Krasnodębska, Kuba Kropielnicki \\ grupa 3a, Numer Indeksu: 325777, 325778}
\newcommand{\titeReport}{Transformacje współrzędnych} % <<< here insert short title in the food
\newcommand{\titleLecture}{Informatyka Geodezyjna II \\ sem. 5, ćwiczenia, rok akad. 2023-2024 \\ Joanna Krasnodębska 325777, Kuba Kropielnicki 325778 grupa 3a}
\newcommand{\kind}{report}
\newcommand{\faculty}{Wydział Geodezji i Kartografii}
\newcommand{\university}{Politechnika Warszawska}
\newcommand{\city}{Warszawa}
\newcommand{\thisyear}{2024}

\pdfinfo
{
	/Title       (GIK PW)
	/Creator     (TeX)
	/Author      (Imię Nazwisko)
}

\begin{document}
	\begin{center} 
		\rule{\textwidth}{.5pt} \\
		\vspace{1.0cm}
		\Large \textsc{\titeReport}
		\vspace{0.5cm} \\  
		\large \textsc{\titleLecture}
		\vspace{0.5cm}\\
		\textsc{\faculty}, \textsc{\university}  \\ 
		\city, \today
	\end{center}

\tableofcontents
\newpage
\section{Wstęp}
\subsection{Cel ćwiczenia}
Celem zadania jest utworzenie skryptu jako klasy, który będzie wykonywał transformacje:
\begin{itemize}

	\item XYZ -> fi, lam, h
	\item fi, lam, h -> XYZ
	\item XYZ -> NEU
	\item fi, lam -> 2000
	\item fi, lam -> 1992
\end{itemize}
\subsection{Wykorzystane narzędzia}
Do napisania programu przez obu twórców zostały wykorzystane:
\begin{itemize}
	\item Python 3.11
	\item Spyder 
	\item System operacyjny: Microsoft Windows 10
\end{itemize}
\section{Etapy rozwiązywania}
Krokiem rozpoczynającym nasze działania było zdefiniowanie klasy 'Transformacje'. Następnie do klasy dodaliśmy algorytmy, pozwalające wykonać transformacje podane w punkcie 1.1. Algorytmy te zostały oparte na funkcjach z ćwiczeń Geodezji Wyższej  semestru 3.
\vspace{0.5 cm}

Następnie przy pomocy \verb|def __init__():|, tworzymy  obiekty klasy Transformacje. Zawierają one informacje o parametrach elipsoidy: a- wielka półoś elipsoidy oraz e2 - mimośród elipsoidy. 
\vspace{0.5 cm}

Następnym krokiem było zaimplementowanie klauzuli \textit{if name == main}, która wykorzystywała "sys.argv". Biblioteka ta umożliwia użytkownikowi przekazywanie argumentów przez wiersz poleceń. Nleży tam podać plik(z którego zostaną wyciągnięte dane), elipsoidę(jedną z opisanych elipsoid), transformacje(jedna z tych opisanych w klasie).
\vspace{0.5 cm}

Po zaimplementowaniu wszytkich transformacji i klauzul, kolejnym etapem było stworzenie plików z przelicoznymi danymi za pomocą komend \verb|open|, \verb|readlines| i \verb|write|.\newline
Końcowym krokiem było uwzględnienie kilku komend i odpowiedzi gdyby użytkownik porgramu nie zaznajmoił się wcześniej z plikiem README.txt. Poniżej przedstawiono kilka przykładów:
\begin{itemize}
	
	\item Jeśli poda się nieistniejącą transformacje dostaniemy powiadomienie: "Transformation function not recognized."
	\item Jeśli poda się dwie sprzeczne ze sobą transformacje to dosatniemy powiadomienie; "Only one flag can be added."
	\item Jeśli poda się model nieuwzględniony w programie to pojawi się wiadomość: "Given model is not supported."
	\item Jeśli poda się model mars z funkcją pl2000 lu pl1992 to pojawi się wiadomość: "Model not supported with this transformation function. Choose a different model."
\end{itemize}

\section{Podsumowanie, wnioski}
\subsection{Link do repozytorium GitHub}
\href{https://github.com/Asiakras/projekt1}{https://github.com/Asiakras/projekt1}

\subsection{Umiejętności nabyte w trakcie rozwiązywania projektu}
\begin{itemize}
	\item Zapoznanie się z obsługą GitHub, gdzie możemy w grupie pracować na jednym pliku i zapisywać zmiany widoczne dla wszytkich
	\item Zapoznanie się ze zmienną sys.argv oraz jej zastosowaniem
	\item Tworzenie dokumentów w latex
\end{itemize}
\subsection{Spostrzeżenia i trudności napotkane w trakcie ćwiczenia}
Pierwszym problemem było skorzystanie z sys.argv, gdyż jest to nowa zmienna i musieliśmy się chwilę dłużej zastanowić jak ona działa i jak skutecznie jej użyć w tym konkretnym ćwiczeniu. Dłuższą chwilę zastanawialiśmy się czemu kod, który wydawał sie poprawny wyrzuca bład: "An exception has occurred, use
 \verb|%tb|  to see the full traceback.
\newline
SystemExit: 1"
\newline
Dopiero później zorientowaliśmy się, że w Spyderze nie będziemy mogli skorzystać z tego polecenia(przez sys.argv), ale w wierszu poleceń wszystko jest sprawne. 
\newline
Użycie i stworzenie komendy \verb|header_lines| wywołało delikatne zamieszanie, ponieważ tworzenie funkcji transformacji i ich flag było podzielone miedzy nami. Samą klauzulę wprowadzono praktycznie pod koniec pisania flag powodując różnicę między kodami twórców.

\section{Bibliografia}
Materiały zajęciowe z ćwiczeń Geodezja Wyższa sem.3 oraz materiały zajęciowe z ćwiczeń Informatyka sem. 4

\end{document}